% !TEX program = xelatex

\documentclass{resume}
\usepackage{zh_CN-Adobefonts_external} % Simplified Chinese Support using external fonts (./fonts/zh_CN-Adobe/)
%\usepackage{zh_CN-Adobefonts_internal} % Simplified Chinese Support using system fonts

\begin{document}
\pagenumbering{gobble} % suppress displaying page number

\name{任宏斌}

\basicInfo{
  \email{hongbinrenscu@outlook.com} \textperiodcentered\ 
  \phone{(+86) 18601282657} \textperiodcentered\ 
  \github{hamletwanttocode}}

\section{\faGraduationCap\ 教育背景}
\datedsubsection{\textbf{中国科学院大学(UCAS)}, 北京, 中国}{2015.09 至今}
\textit{博士}\, 理论与计算物理, 计划毕业时间 2021.06
\datedsubsection{\textbf{台湾大学(NTU)}, 台北, 中国}{2013.09 -- 2014.02}
\textit{交换生}\, 物理学系
\datedsubsection{\textbf{四川大学(SCU)}, 成都, 中国}{2011.09 -- 2015.06}
\textit{学士}\, 物理学

\section{\faUsers\ 科研经历}
\datedsubsection{\textbf{中科院物理研究所}, 北京, 中国}{2015.09 至今}
\role{科研助理}{导师: 王磊研究员, 戴希教授}
\begin{itemize}
  \item 量子系统动能泛函的机器学习研究
  \begin{itemize}
    \item 使用\textit{PCA}算法对数据进行特征提取,并通过高斯过程及其关于输入变量的导数精确学习一维量子系统的动能及动能导数
    \item 独立开发了利用自动微分求高斯过程中超参数导数的程序包\textit{GPFlux.jl}
    \item 通过使用基于梯度的训练方法将此前的研究中同类算法的训练时间缩短了\textbf{80\%}
  \end{itemize}
  \item 科学机器学习及其在量子化学中的应用
  \begin{itemize}
    \item 设计了基于密度矩阵的全变分哈特里-福克算法,使用\textit{BFGS}优化算法替代了传统的自洽算法,从理论上提高了算法的收敛性
    \item 独立开发了可微分量子力学编程库\textit{QuantumLang.jl},并利用它实现了我们提出的算法
    \item 同等精度下,我们的实现的算法在各项基准测试中比之前发表的算法有\textbf{100}倍效率提升
  \end{itemize}
\end{itemize}

\section{\faUsers\ 社会实践}
\datedsubsection{\textbf{北京高思教育集团}, 北京, 中国}{2016.04 -- 2018.06}
\role{高中物理教师}{}
\begin{itemize}
  \item 所带班级秋冬升班率可达到\textbf{80\%},在公司所有高中物理老师中排名前\textbf{5\%}
  \item 通过个性化的辅导及同学生家长的有效沟通,使得班上学生物理学科期中/期末考试成绩有明显提升
  \item 在原有基础上,自主改进出一套高中物理授课讲义
\end{itemize}

\section{\faCogs\ 编程及开源贡献}
\begin{itemize}[parsep=0.5ex]
  \item \textbf{语言:} Python, Julia
  \item \textbf{开源项目:} 自主编写了\textit{GPFlux.jl}, \textit{QuantumLang.jl};并向\textit{Stheno.jl} (2 merged PRs), \textit{NiLang.jl} (1 merged PR) 贡献代码
\end{itemize}

\section{\faHeartO\ 所获奖励}
\datedline{中科院物理所所长表彰奖}{2017}
\datedline{四川省优秀毕业生}{2015}
\datedline{四川大学校级一等奖学金}{2013}

\end{document}
